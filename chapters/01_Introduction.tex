\chapter{Introduction}

\say{I am a HAL 9000 computer. I became operational at the H.A.L. plant in Urbana, Illinois on the 12th of January 1992. My instructor
was Mr. Langley, and he taught me to sing a song.}

These words were spoken by HAL 9000, the artificial general intelligence depicted in the movie \textit{2001: A Space Odyssey}
by Stanley Kubrick, published back in 1968. In this film, HAL 9000 is in charge of controlling the systems of the
\textit{Discovery One} spacecraft and interacting with the ship's astronaut crew. The abilities of this computer were impressive: it
was capable of speech recognition, facial recognition, natural language processing, automated reasoning and many other features
characteristic of the most complete artificial intelligence ever created. And on top of that, it was also capable of doing tasks
that are now known as home automation.

Of course, in 1968 the field of Artificial Intelligence was only taking its first steps, and these features were only a dream in many
people's minds.\cite{harvardHistory} Nevertheless, \textit{2001: A Space Odyssey} contributed greatly to the popularization of 
these technologies among the general public and, most notably, familiarized people with new paradigms of human computer interaction.

Today, home automation and voice assistance are experiencing one of their most popular moments, thanks to the lower cost of components
and the incredible development of Artificial Intelligence and Internet of Things by leading companies. And most importantly, these
long-awaited technologies are finally within everyone's reach.

\section{Motivation}
The interest from companies about home automation and voice assistance has been growing in this decade. Currently, we can find
solutions from technology companies that combine a virtual assistant with a home automation system, and these are exactly the most
popular ones.\cite{techRadarBest}

On the other hand, companies that have classically made home appliances and lightning systems, like Philips, are now entering the smart 
home market.\cite{philipsHueMeethue} The range of \textit{smart devices} is enormous at the moment, and many users may feel lost when
looking for a solution for their homes. This is one of the problems that we identified, but not the only one.

Another big problem is that home automation products tend to work only with other devices from the same maker. For example, Philips
lightning systems require a Philips bridge and a Philips mobile application in order to work. But if the user has lights from different
makers, he will probably need to install more bridges and more applications in his mobile phone. However, all bridges usually do the same
job: receiving commands via WiFi or cable and sending them to the domotic devices via Zigbee or Z-Wave, for example (both are popular
communication protocols in domotics). Makers are not moving towards unification, but to differentiation.

Luckily, there are some systems that can unify a bit a home automation system composed by devices from different makers. For example,
Apple HomeKit\cite{appleIOSHome} or Amazon Alexa\cite{amazonAlexa}. However, these products are usually expensive and their 
customization is very limited. They also fall short of availability, as these previous devices are not yet available in Spain, nor 
in a large number of countries.

\bigskip
\section{Objectives}
From the previous section, we can see the need for having an affordable and customizable home automation system that can group 
devices from different manufacturers, and that offers as many facilities as the aforementioned systems. The main objective in this 
project is to create a home automation controller based on new human computer interaction paradigms, such as the voice, that meets
these requirements.

This system needs to be modular, extensible, safe and fully customizable. It is desirable to include automation capabilities on it, 
as well as accessibility from a mobile application. The common processes, such as discovering domotic devices in the network and
adding them, must be seamless and easy, as in the current home automation systems.

\subsection{Specific Objectives}
\begin{enumerate}
	\item Integrate a home automation controller in an embedded system, like a Raspberry Pi.
	\item Integrate a voice assistant in the same embedded system as the home automation controller.
	\item Explore current home automation systems and voice assistants, focusing on open-source solutions.
	\item Explore how we can express automation rules to manage home devices.
	\item Explore options for managing the system from a mobile application.
	\item Explore options for providing everywhere access to the system.
	\item Explore safety and privacy concerns related to the home automation system.
	\item Provide an adaptive and responsive user interface, usable on touch and non-touch screens.
	\item Connect the virtual assistant to the domotic system to manage directly the configured devices.
	\item Develop the home automation system applying usability techniques in order to obtain an intuitive and functional solution.
\end{enumerate}

\bigskip
\section{Structure of the Work}
This work is structured in eight chapters and two appendices. The objective is to introduce first all the results of our research, 
that is, the general and specific concepts and the most important products related to this project to provide a knowledge base 
in order to better understand the development of the resultant project.

Chapter 2 explains the management of this project and the methodologies that we have followed. The tools used in order to keep track
of the tasks and the time spent, as well as the documentary management, will be briefly introduced in this chapter.

Chapter 3 introduces the home automation technology. In this chapter, we explore the different concepts of home automation,
its main features and its history. We also provide data and statistics about the attitude of society towards this technology. Then,
we focus on home automation system design, indicating the different possible architectures that a domotic system can have.

Chapter 4 is about voice assistance, another very important part in this project. The objective is similar to the previous
chapter, we explain what are the virtual assistants and, more precisely, the voice assistance technology. We give examples of where
we can find virtual assistants and, in the second section of this chapter, we indicate the capabilities and services that virtual
assistants can provide.

In Chapter 5, we analyze many different products related to this project. It is divided in three sections: home automation
systems, home automation devices and voice assistants. In the first section, we explore the most popular home automation systems
on the market, as well as other open source software. In the second one, we explore home automation devices that are made for very
different purposes, classify them by type and indicate the pros and cons for each one, and also regarding their integration with openHAB.
In the third section, we do the same for voice assistants, exploring the main systems currently in the market. For the home automation
systems and for the voice assistants, we end their sections with a comparative table of all the options we have presented.

Chapter 6 offers a deeper insight into openHAB, a home automation system previously presented in Chapter 5. OpenHAB is a
huge system worth exploring in depth, and in this chapter we introduce it, as well as its history and structure. We then explore its
main concepts from a \textit{logical} point of view, and next we offer a developer's perspective, explaining the internal organization
of the software, its installation and other technical concepts.

We explain the entire development process of the Voice-Driven Home Automation Controller in Chapter 7. First, we provide an analysis 
of the system from a software engineering perspective, indicating product specification, system analysis and system design. Then, 
we describe the implementation process from the installation of the system. This chapter is mainly technical, includes snippets of 
code that we have used in the project. Also, the appendix A is related to this chapter, where the full script that composes the 
voice assistant is included.

This work ends with Chapter 8. Here we analyze the obtained results and give ideas for future developments based on this
work. We also analyze the fulfillment of the specific objectives specified in this chapter.

