\chapter{Voice Assistance}

We spend so much time using devices that have integrated voice assistants that we usually forget how incredibly fast they have
evolved. Nowadays, they can recognize thousands of words and expressions really fast, and they are even capable to imitate
emotions. What is more, they fit in a pocket. But the reality was totally different just a couple of decades ago. From the IBM
Shoebox to Siri, in this chapter I will explore the fundamentals of voice assistance.

\section{What is voice assistance?}
Voice assistance is the result of another form of interaction between humans and computers.\cite{botsocietyVUI} The Voice User
Interface (VUI), which has the voice assistants as a result, allows a user to interact with computer or mobile or other electronic 
devices through speech or voice commands. Thus, it is an interface of any speech recognition applications.

Then, a voice assistance, also known as virtual assistant, is an application program that understands natural language voice commands
and can perform tasks or services for an individual.
