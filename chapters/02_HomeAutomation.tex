\chapter{Home Automation}

Home automation, also known as domotics, has been a recurrent topic in Computer Science that
has become a reality in the last decades, thanks to the growth and decrease in the price of embedded
systems and wireless technologies, that have permitted to create distributed systems, the heart of this technology.

\bigskip
In this chapter, I am going to analyze this technology and its current state, including its implementation in commercial
products.

\section{What is home automation?}

Although science fiction has represented the idea of smart houses since the past century, including in them
an intelligence able to respond to all the dweller’s needs and desires, it has never felt as close to real world as today.

\bigskip
The basic idea of home automation is to employ sensors and control systems to monitor a dwelling, and accordingly 
adjust the various mechanisms that provide heat, ventilation, lighting, and other services. By more closely tuning the 
dwelling’s mechanical systems to the dweller’s needs, the automated \"intelligent\" home can provide a safer, more 
comfortable, and more economical dwelling.\cite{smarthouse98} For example, the automated system can determine 
the intensity and direction of the sunlight, and adequate the house according to its condition (which would include
closing the blinds and adjusting the air conditioner).

\bigskip
Unlike many may think, we don't actually need a very modern house, since advanced systems can be perfectly integrated 
in older, traditional buildings. This fact makes domotics a real possibility in every situation. In fact, the number of home 
automation systems installed in Europe is expected to reach around 29 million by 2019.\cite{statistaInstalled}

\begin{figure}
	\centering
	\includegraphics[width=0.9\textwidth]{images/Chapter_02/security.jpg}
	\caption{Example of a smart home with security-oriented devices}
	\label{fig:security-in-smarthome}
\end{figure}

\bigskip
There is not an exact point where we can set the beginning of the domotics as a real concept, but during the last century
there has been some remarkable efforts, and even before. In 1898, Nikola Tesla created a wireless control for a toy boat, 
the first of its kind \cite{betanewsHistory}. That marks the beginning of wireless technologies, one of the fundamental
parts of Home Automation. 

\bigskip
In 1975, after lots of appearances of the idea of home automation in films, the first general purpose home automation
technology, called X10, was developed. X10 defines a protocol for communication between electrical devices, which uses power
line wiring for signaling and control, where the signals involve brief radio frequency bursts representing digital information.
Therefore, it also defines a wireless radio based protocol. Surprisingly, the X10 technology is still widely used and available,
with millions of units in use worldwide.

\bigskip
However, it was not until 1984 that the word Smart Home appeared, invented by the \textit{American Association of House
Builders}. After that, different inventions rapidly followed one another, with devices such as\\ \textit{The Clapper} (which 
was operated through sound, like a clap or a bark) and interest from the biggest technological companies, like Microsoft.

\begin{figure}
	\centering
	\includegraphics[width=0.5\textwidth]{images/Chapter_02/the-clapper.jpg}
	\caption{The Clapper, a sound-activated switch}
	\label{fig:the-clapper}
\end{figure}

\bigskip
Home Automation has not stopped gaining ground on our homes and now it is experiencing one of the best moments
in its lifetime, with the unstoppable growth of the Internet of Things (IoT) and the simultaneous development of Artificial 
Intelligence for the general public, with the biggest companies, like Google and Apple, investing millions of dollars on it.
Devices like Amazon Echo and Google Home, or assistants like Siri, Cortana, Google Assistant and Amazon Alexa are a 
good representative of this trend. I will talk in depth about them in the following sections.

\bigskip
We have always imagined that Smart Homes would bring us a whole world of benefits. And that is partly true, but
they have ended up offering benefits that no one could imagine some decades before, when matters such as energy
savings were not as important as today. These benefits are responsible for their increasing popularity, and they can be 
summarized in the following points:

\begin{itemize}
	\item \textbf{Control anywhere:} Smart Homes can be completely controlled anywhere in the world from smart phones or
	other devices with Internet connection, so we can know the status of our devices at any time. That would allow us, for
	example, to stop worrying when staging out of home thinking if we have left the air conditioning on.
	\item \textbf{Safety:} there are tons of security systems ready to work on Smart Houses. They are capable of monitoring
	the people going in and out of home and send alerts to the owners if necessary. Like many other devices, there are also
	smart locks for the door and cameras that we can control from our smart device.
	\item \textbf{Accessibility:} Smart Homes can increase a lot the quality of life of elderly or disabled people, as they can be
	managed via voice commands, making the interaction much easier to people which is not experienced with computers and
	improving their independence.
	\item \textbf{Energy efficiency:} one of the main goals of Home Automation is to work with the least amount of energy needed,
	and a big part of the research in this field is going in this direction. There are induction cook-top stoves that can be powered on
	only if there is anything placed over them (and even get the perfect cooking, powering off themselves)\cite{directenergyAdvantages}
	or heating systems that power on and off depending on the weather and inner conditions of the home, or even a faucet technology
	that can maximize shower water usage by shaping the individual droplets of water, so the experience feels almost the same but with
	less water usage. 
	\item \textbf{Money saving:} the last point leads to another benefit: saving money. Smart Homes can use less energy and water,
	making a big difference in how much we pay at the end of the month. Reports show that the savings on the energy bill for this 
	reason range from 10\% to 30\%.\cite{directenergyAdvantages}
	\item \textbf{Confort:} Smart Houses can also help save time. Today, when everyone is trying to make the most of their free time, 
	this technology is capable of doing housework, so that people can spend their time on things they enjoy most, or simply gain time 
	to spend with their families.
\end{itemize}

\bigskip
This range of benefits has made possible to see home automation systems in many homes, but also in offices. Now, almost every new
house that is built is prepared for domotics, including Internet access points in every room, a big amount of plugs, and a lot of space 
to extend its capabilities in a future. Indeed, the global home automation and security control market is expected to reach 12.81 billion
dollars by 2020.



