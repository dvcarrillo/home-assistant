\chapter{Conclusions and future work}

Since the beginning of this project, the main objective has been maintained: to create an affordable, functional and usable voice-driven
home automation controller. At this point, we can tell that this objective has been mostly reached, although there are many things
to improve in a future.

When I began working on this system, I only knew some the most popular commercial virtual assistants with a home automation system,
like Google Home or Amazon Alexa, and none of them had been launched in the Spanish market. Today, almost ten months after, only
Google Home is available. It was when I started this project that I realized how many open source solutions were available, like openHAB.
This allowed me to work over an existing basis and to improve its capabilities, always maintaining the main objective of affordability
and functionality. The end result, although much less flexible, can offer most of the features offered by the main products on the
market. The custom voice assistant, that uses Google Assistant to answer all the commands not related to openHAB, makes this product
very useful for all kinds of situations.

One of the main issues that we have found is that this system needs to be configured by a technical person at least in its installation.
The installation of openHAB itself requires some technical knowledge, but its configuration and, most importantly, the configuration
of the voice assistant, requires coding. A very convenient future improvement would be to automate the generation of the voice
assistant script every time that a device is added. Of course, it would also be a good improvement to automate all the process.

This project relies heavily on third party solutions, like the Google services for processing the speech or the service
\textit{myopenHAB}. However, we have tried to explore and provide alternatives in every case, that would cover any different situation.

Another important future improvement is related to privacy. We observed that many devices, such as the Philips Hue Smart Bridge,
are in constant communication with the cloud, when in many cases this is not necessary. Some way of blocking this communication
and restricting it to the local network should be explored.

\begin{table}[]
	\centering
	\resizebox{\textwidth}{!}{%
		\begin{tabular}{|l|c|l|}
			\hline
			\multicolumn{1}{|c|}{\textbf{Objective}} & \textbf{Fulfillment} & \multicolumn{1}{c|}{\textbf{Observations}} \\ \hline
			\begin{tabular}[c]{@{}l@{}}1. Integrate a home automation \\ system in a embedded system\end{tabular} & 100\% & \begin{tabular}[c]{@{}l@{}}We have successfully installed and integrated\\ openHAB in the Raspberry Pi\end{tabular} \\ \hline
			\begin{tabular}[c]{@{}l@{}}2. Integrate a voice assistant in \\ the same embedded system as \\ the home automation system\end{tabular} & 100\% & \begin{tabular}[c]{@{}l@{}}The custom assistant integrates Google\\ Assistant and a voice assistant for\\ openHAB\end{tabular} \\ \hline
			\begin{tabular}[c]{@{}l@{}}3. Explore current home \\ automation systems and \\ voice assistants, focusing \\ on open-source solutions\end{tabular} & 80\% & \begin{tabular}[c]{@{}l@{}}We have explored some of these systems on\\ Chapter 3, but the inclusion of open source\\ solutions has been reduced\end{tabular} \\ \hline
			\begin{tabular}[c]{@{}l@{}}4. Explore automation \\ possibilities and implement\\ an automation service in the \\ domotic system\end{tabular} & 70\% & \begin{tabular}[c]{@{}l@{}}We have successfully implemented an \\ automation service using IFTTT and \\ the REST API of openHAB. Some other \\ services might have been explored\end{tabular} \\ \hline
			\begin{tabular}[c]{@{}l@{}}5. Explore options for \\ managing the system from a \\ mobile application\end{tabular} & 90\% & \begin{tabular}[c]{@{}l@{}}The system is currently manageable\\ thanks to the Cloud Connector and the\\ mobile application of openHAB\end{tabular} \\ \hline
			\begin{tabular}[c]{@{}l@{}}6. Explore options for \\ providing global access to \\ the system and implement one\end{tabular} & 100\% & \begin{tabular}[c]{@{}l@{}}We have explored these options on Chapter 6 \\ and we have tested the service \\ myopenHAB\end{tabular} \\ \hline
			\begin{tabular}[c]{@{}l@{}}7. Explore safety and privacy \\ concerns related to the home \\ automation system\end{tabular} & 30\% & \begin{tabular}[c]{@{}l@{}}Some safety concerns have been covered\\ through this work, but we have not covered\\ the main aspects related to privacy\end{tabular} \\ \hline
			\begin{tabular}[c]{@{}l@{}}8. Provide an adaptive and \\ responsive user interface, \\ usable on touch and non-touch \\ screens\end{tabular} & 100\% & \begin{tabular}[c]{@{}l@{}}We explained how to configure Basic UI and\\ PaperUI. Both user interfaces comply \\ with these points\end{tabular} \\ \hline
			\begin{tabular}[c]{@{}l@{}}9. Connect the virtual assistant\\ to openHAB, so it can manage \\ the devices present in the \\ system\end{tabular} & 100\% & \begin{tabular}[c]{@{}l@{}}The custom virtual assistant communicates\\ with openHAB thanks to the REST API\\ of openHAB and the Requests Python \\ library\end{tabular} \\ \hline
			\begin{tabular}[c]{@{}l@{}}10. Test domotic devices in the\\ final system and present an \\ usable solution\end{tabular} & 70\% & \begin{tabular}[c]{@{}l@{}}The current solution is completely\\ functional and usable. However, more \\ devices should have been tested\end{tabular} \\ \hline
		\end{tabular}%
	}
	\caption{Fulfillment of the specific objectives presented in Chapter 1}
	\label{table:fulfillment-objectives}
\end{table}

The table \ref{table:fulfillment-objectives} presents the fulfillment level of the specific objectives that we presented in the
Introduction chapter.
