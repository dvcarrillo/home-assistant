\chapter{Product Analysis}

The aim of this chapter is to provide a detailed analysis of the devices more closely related to this project, now that we have a 
clearer idea about its main pillars. I will go into many of the available commercial devices and software in the fields of home 
automation, voice assistance and smart devices.

\section{Home Automation Systems}
This section covers all the hardware and software systems related to home automation. As we will see, there are lots of solutions
with very different purposes: while there is Amazon Alexa, a full hardware and software system that integrates other home automation
systems, we can also find pure online solutions, like the automation platform IFTTT. Sometimes, home automation systems are
built underneath a virtual assistant, as it happens with Amazon Alexa, so some devices are going to appear in this section and in the
next one. However, they will analyzed from two different perspectives, as having a good virtual assistant does not mean having a
good home automation system.

\subsection{Philips Hue}
Philips Hue is a personal wireless lighting system aimed at the smart home. In combines LED light bulbs, LED strips and other 
lightning devices, and sensors that can be configured in their mobile app, so they can modify the home lightning based on a set of
rules. There is a wide range of products, including color and only white lights, so users can build a pretty customizable lightning
experience.\cite{philipsHueMeethue}

The system requires a bridge connected to the Internet (called Philips Hue Smart Hub) in order to work. This is because the Hue
devices do not use WiFi in order to communicate with the bridge, but the system needs to have WiFi to be controllable from a 
mobile phone. So, it follows a centralized architecture. Moreover, Philips does not provide any type of assistant or external interface
to manage the system apart from the mobile application by default, although Hue works with the most popular home automation
systems, like Alexa or Apple Home, that provide much more flexible home automation management.

\subsection{LG SmartThinQ}
LG SmartThinQ groups the range of Wi-Fi enabled home appliances made by the company LG, including refrigerators, dishwashers,
vacuum cleaners or air purifiers, between others. As of September 2017, they were the most extensive range of devices of their kind.\cite{lgSmartThinq}

Unlike Philips Hue, SmartThinQ devices do not require a bridge to work. They can be controlled from the mobile phone and, in some
cases, like in the refrigerators, they include a touchscreen to interact with the device. However, LG does not provide any extra device
or virtual assistant to interact with them, though they are manageable through Amazon Alexa and Google Assistant.




% Incluir dispositivos que he estudiado, sistemas domóticos y alternativas a openHAB, asistentes de voz, etc
% Ejemplo de comparación entre asistentes de voz: https://en.wikipedia.org/wiki/Virtual_assistant