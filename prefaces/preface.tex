\chapter*{}
\thispagestyle{empty}
\cleardoublepage

\thispagestyle{empty}

\begin{titlepage}
 
 
\setlength{\centeroffset}{-0.5\oddsidemargin}
\addtolength{\centeroffset}{0.5\evensidemargin}
\thispagestyle{empty}

\noindent\hspace*{\centeroffset}\begin{minipage}{\textwidth}

\centering
%\includegraphics[width=0.9\textwidth]{imagenes/logo_ugr.jpg}\\[1.4cm]

%\textsc{ \Large PROYECTO FIN DE CARRERA\\[0.2cm]}
%\textsc{ INGENIERÍA EN INFORMÁTICA}\\[1cm]
% Upper part of the page
% 

\vspace{3.3cm}

%si el proyecto tiene logo poner aquí
% \includegraphics{images/logo.png} 
\vspace{0.5cm}

% Title

{\huge\bfseries Creation of a voice-driven controller for home automation\\
}
\vspace{0.5cm}
\end{minipage}

\vspace{2.5cm}
\noindent\hspace*{\centeroffset}\begin{minipage}{\textwidth}
\centering

\textbf{Autor}\\ {David Vargas Carrillo}\\[2.5ex]
\textbf{Director}\\
{Juan Antonio Holgado Terriza}\\[2cm]
%\includegraphics[width=0.15\textwidth]{imagenes/tstc.png}\\[0.1cm]
%\textsc{Departamento de Teoría de la Señal, Telemática y Comunicaciones}\\
%\textsc{---}\\
%Granada, mes de 201
\end{minipage}
%\addtolength{\textwidth}{\centeroffset}
\vspace{\stretch{2}}

 
\end{titlepage}




\cleardoublepage
\thispagestyle{empty}

\begin{center}
{\large\bfseries \myTitleES}\\
\end{center}
\begin{center}
\myName\\
\end{center}

%\vspace{0.7cm}
\noindent{\textbf{Palabras clave}: domótica, asistencia por voz, sistemas distribuidos, Raspberry Pi, software libre}\\

\vspace{0.7cm}
\noindent{\textbf{Resumen}}\\

El objetivo principal de este proyecto es la creación de un controlador domótico activado por voz en un sistema embebido,
como la \textit{Raspberry Pi}, centrándose en el uso de software libre, obteniendo la máxima compatibilidad y el mínimo
coste.

\bigskip
Para conseguirlo, se ha analizado la situación actual del sector, distinguiendo entre dispositivos domóticos,
asistentes de voz y sistemas orientados a la automatización del hogar. A través de la Ingeniería del Software, se han estudiado
las posibles necesidades de los usuarios, intentando suplir las carencias actuales del sector. Finalmente, se presenta una
implementación de un sistema domótico en un entorno real, utilizable y extensible a cualquier situación cotidiana.

\bigskip
Por tanto, el proyecto trata de demostrar las infinitas oportunidades que habilita el reciente campo de la domótica, y la 
posibilidad de crear sistemas domóticos funcionales de bajo coste.

\cleardoublepage


\thispagestyle{empty}


\begin{center}
{\large\bfseries \myTitle}\\
\end{center}
\begin{center}
\myName\\
\end{center}

%\vspace{0.7cm}
\noindent{\textbf{Keywords}: home automation, voice assistance, distributed systems, Raspberry Pi, open source}\\

\vspace{0.7cm}
\noindent{\textbf{Abstract}}\\

The main goal of this project is the creation of a low-cost, voice-driven home automation controller in a embedded system, such 
as the \textit{Raspberry Pi}, using open source technologies and trying to obtain maximum compatibility with minimum cost.

\bigskip
To achieve this, I have analyzed the current state of the sector, distinguishing between domotic devices, voice assistants
and home automation oriented systems. Through Software Engineering, I have studied the possible necessities of the users, trying
to make up for the scarcities in this sector. Finally, I show an implementation of a home automation system in a real environment,
usable and extensible to any daily situation.

\bigskip
Therefore, this project tries to demonstrate the infinite opportunities that the recent field of domotics enables, and the possibility
of creating low-cost functional home automation systems. 

\chapter*{}
\thispagestyle{empty}

\noindent\rule[-1ex]{\textwidth}{2pt}\\[4.5ex]

Yo, \textbf{\myName}, alumno de la titulación GRADO EN INGENIERÍA INFORMÁTICA de la \textbf{\myFaculty de la \myUni}, 
con DNI 76592492P, autorizo la ubicación de la siguiente copia de mi Trabajo Fin de Grado en la biblioteca del centro 
para que pueda ser consultada por las personas que lo deseen.

\vspace{6cm}

\noindent Fdo: \myName

\vspace{2cm}

\begin{flushright}
\myLocation, a \myTimeES
\end{flushright}


\chapter*{}
\thispagestyle{empty}

\noindent\rule[-1ex]{\textwidth}{2pt}\\[4.5ex]

D. \textbf{\myProf}, Profesor del \textbf{\myDepartment} de la \textbf{\myUni}.

\vspace{0.5cm}

\textbf{Informa:}

\vspace{0.5cm}

Que el presente trabajo, titulado \textit{\textbf{\myTitle}}, ha sido realizado bajo su supervisión 
por \textbf{\myName}, y autoriza la defensa de dicho trabajo ante el tribunal que corresponda.

\vspace{0.5cm}

Y para que conste, expide y firma el presente informe en \myLocation, a \myTimeES.

\vspace{1cm}

\textbf{El director:}

\vspace{5cm}

\noindent \textbf{\myProf\\}

\chapter*{Agradecimientos}
\thispagestyle{empty}

\vspace{1cm}

\hspace{4ex}A mis padres, cuyo esfuerzo y dedicación han hecho que hoy esté escribiendo estas líneas.

\bigskip
A todos los compañeros y amigos que han estado conmigo en este camino, por haberlo hecho 
mucho más agradable y ameno.

\bigskip
Y, por supuesto, a Juan Antonio, por haber aceptado mi idea y haber hecho posible este proyecto.
